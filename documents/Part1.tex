\section{Preliminaries}
\label{sec:preliminaries}

\subsection{The problem}
In this section, we introduce our notations and formalize Subgroup
Discovery. We introduce two ``flavours'' of the problem. In
the first variant, we seek the \emph{k best} subgroups. In the second one,
we seek \emph{k diverse} sugroups.

\subsubsection{Definitions}
Consider a database $DB$. We use the expressions \emph{variable} and
\emph{column} indifferently. The set $\Gamma$
represents the $|\Gamma|$ non-target variables, and $T$ represents the target
variable.

Our aim is to discover interesting sugroups. A subgroup is a set of tuples $G
\subset DB$, defined by a \emph{description} $d(G)$. In principle, the
description is expressed in a language chosen by the user. In this paper, we
restrict the language to conjunctions (e.g., $\text{age} > 40 \wedge
\text{size} < 180$). We denote $\gamma(G)$ the variables which appear in $G$'s
description (e.g., \{age, size\}).

To evaluate a subgroup, we need a \emph{quality measure}. A quality measure is
a function $\varphi : L \to \mathbb{R}$, where $L$ is the language. This
function assigns a score to each subgroup.  By convention, higher is better.

\subsubsection{Objective}
We can now define our two problems. The first problem describes the ``usual'' Subgroup
Discovery task:

\begin{problem}
    (Top-k subgroup discovery) Given a database $DB$, a quality measure
    $\varphi$ and an integer $k$, find the $k$ best subgroups with regards to
    $\varphi$.
\end{problem}

This approach is based on local criteria: the aim is to maximize the quality
for each subgroup separately. As pointed out by out  \cite{van2011non}, this
approach tends to produce highly redundant patterns sets. To avoid this, we use
a \emph{global} approach: 

\begin{problem}
    (Diverse-k subgroup discovery) Given a database $DB$, a quality measure
    $\varphi$ and an integer $k$, find a non redundant set of $k$ high quality subgroups
\end{problem}

As the authors of the original paper point out, defining redundancy 
is non trivial. In our work, we consider that two subgroup are redundant when
their \emph{description} overlap. [there are other definitions in the paper]

\subsection{Quality measures}
In this section, we describe a few usual quality measures. This includes (but
is not limited to ) the Weighted Relative Accuracy (WRAcc), the Weighted
Kullback-Leibler divergence (WKL), the weighted Krimp gain (WKG).

\subsection{Search strategies}
In this section, we explain beam search and the alternatives.
