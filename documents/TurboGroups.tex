\documentclass{sig-alternate}

\usepackage{graphicx}
\usepackage{caption}
\usepackage{subcaption}
\usepackage{balance}
\usepackage{algpseudocode}
\usepackage{algorithm}
\usepackage{enumitem}

\usepackage{color}
\usepackage{tabulary}
\usepackage[table]{xcolor}
%\definecolor{red}{RGB}{242,175,173}
%\definecolor{grn}{RGB}{217,228,170}

\newtheorem{definition}{Definition}
\newtheorem{problem}{Problem}
\newtheorem{lemma}{Lemma}

\newcommand{\rv}{\ensuremath{\mathcal}}


\begin{document}

\title{Semi-Automated Exploration of Data Warehouses}


\numberofauthors{3}
\author{
\alignauthor
Thibault Sellam\\
       \affaddr{CWI}\\
       \email{thibault.sellam@cwi.nl}
\alignauthor
Emmanuel M\"uller\\
       \affaddr{KIT}\\
       \email{emmanuel.mueller@kit.edu}
\alignauthor
Martin Kersten\\
       \affaddr{CWI}\\
       \email{martin.kersten@cwi.nl}
}

\maketitle

\begin{abstract}
Exploratory data analysis tries to discover novel dependencies and unexpected
patterns in large databases. Traditionally, this task is manual and
hypothesis-driven. However, analysts can quickly come short of patience and
imagination. In this paper, we study how to detect interesting views and
selections automatically, exploiting non-linear statistical dependencies across
the columns of the database. We introduce simple notions of information theory,
and derive a complete model of interestingness. We then present Claude, our
hypothesis generator for data warehouses. Claude follows a 2-step approach: (1)
It detects interesting views with a level-wise, greedy strategy.  (2) To
explain its findings, it detects local patterns in these views and describes
them with SQL queries.  To implement Claude, we developped aggressive
approximations and heuristics, allowing our system to be both fast and more
accurate than state-of-art view search algorithms. 
\end{abstract}
\section{Introduction}
\label{sec:intro}

{ \color{red}
    Emphasize the hypothesis generation, remove references to OLAP}

Consider a data warehouse scenario: a database describes how a set of
\emph{measures} varies along several \emph{dimensions}. To work with this
database, analysts need a lot of prior knowledge: which dimensions and measures
they are interested in, where interesting patterns are hidden in the cube,
which dependencies among dimensions and measures exist, and much more.  Manual
exploration may expose some of this knowledge, but its success depends entirely
on chance and intuition. In contrast, we aim at a semi-automated exploration.
We want to assist the humans in understanding how measures are distributed,
where unexpected measures appear, and where interesting dependencies between
dimensions and measures arise.

Semi-automated exploration of data cubes is fundamentally challenging for two
reasons. The first reason is obvious: how do we recognize ``interesting''
queries?  The diversity of opinions in the literature is rather depressing.
What is interesting for a user can be extremely boring for another. The second
problem is practical.  Suppose that we had a universal measure of interest, how
could we explore the search space fast enough to find the interesting queries?

Many authors have proposed semi-automated (or discovery-driven) exploration
frameworks in the past. Typically, they introduce a fixed ``interestingness''
model, then the exploit the particularities of their model to speed up
computations.  For instance, a seminal work was presented by Sarawagi et al.
\cite{sarawagi1998discovery}. According to their paper, the most interesting
queries are the most ``surprising'' ones.  They suggest to build a (log-linear)
model over the data, and identify the largest deviations. More recently, Dash
et al.\cite{dash2008dynamic} proposed a facet selection method, also based on
surprise. Nevertheless, given the diversity of users and requirements, are
these interestingness models \emph{really} interesting?

In this paper, we introduce \textit{Claude}, a generic query generation
system. \emph{Claude lets users describe what they are looking for.
Given their input, it explores the database and reveals interesting OLAP views}.
Here is how it operates. First, the users assign a value to each tuple in
the database.  We call this value the \emph{target}. It can come from the raw
data (e.g., the brightness of light sources), a calculation (e.g., a
``surprise'' score from the litterature), or manual effort. Claude's job is to
present the ``big picture'': it explains how the target is distributed across
the database, and what influences it. To do so, it operates in two steps.
First, it explores the database for interesting projections. Then, for each
projection, it reveals ``remarkable'' areas. 

Making Claude efficient is a major challenge. We
introduce two algorithms.  The first one is exact and based on a systematic
level-wise search paradigm.  The second one is a much faster and based on a
relaxation of our model and a heuristic search.

\section{Presentation and Objectives}

\subsection{Overview}
\begin{figure}[t!]
\centering
\includegraphics[width=\columnwidth]{images/Overview}
\caption{Example of explanation, with two points of interest. All the values
are normalized.}
\label{fig:overview}
\end{figure}
We represent the database by a large table $DB$. This table contains two types
of columns: the dimensions $X_0, \ldots, X_N$ and a target column $T$. This
view is logical, we are oblivious to the physical structure of the data.
Claude's aim is to describe what causes the target to vary. To do so, it
generates \textbf{explanations}.  An explanation is a set of two elements: a
\textbf{view} and a list of \textbf{points of interest} (POI). A view is an
``interesting'' set of columns.  More exactly, it is a set of columns which
have a strong \emph{influence} on the target. A POI is a region where this
influence is maximal, and thus the target takes unusual values.

We illustrate these notions with a running example. We want to understand what
causes crime in US communities. We have a database of cities with several dozen
socio-economic indicators (for instance, employment, age, or diplomas). For
each city we also have the annual number of violent crimes: this is our target
variable. This example comes from a real use case, which we will
present in Section \ref{sec:crime}.

Figure~\ref{fig:overview} presents an example of explanation. The leftmost
scatterplot is a view based on the dimensions \texttt{Unemployment} and
\texttt{Population Density}. Why did Claude pick these two columns? The two
POIs provide more explanations. Cities with high densities and high
unemployment rates have more crimes (cf. $POI_1$). Oppositely, cities which low
densities and lower emplyments rates are safer (cf. $POI_2$). We see that the
dimensions \texttt{Unemployment} and \texttt{Population Density} influence our
target variable, \texttt{Crime}, and that this influence is maximal in the
POIs.

In practice, Claude expresses its recommendations with SQL queries. It returns
points of interest as follows:
\begin{verbatim}
    SELECT X1, ... , Xd, T
    FROM DB
    WHERE X1 BETWEEN [L1, H1]
      AND ... 
      AND Xd BETWEEN [Ld, Hd]
\end{verbatim}
In this query, $\texttt{X1},\ldots, \texttt{Xd}$ represent the variables of the
view, $\texttt{[L1, H1]}\ldots\texttt{[Ln, Hn]}$ represents the bounds of the
POI, and $T$ represents the target.

\subsection{Formalization}
\label{sec:formalization}
We introduced our notions of explanations, views and POIs. We now formalize
these concepts with elements of information theory~\cite{cover2012elements}. In
the remaining of this paper, we suppose that each column $X_n = (x_n^1, \ldots,
x_n^M)$ contains $M$ samples drawn from a random variable $\rv{X}_n$ with
sample space $\Omega_n$.

\subsubsection{View Strength}
\label{sec:view-strength}
First, let's define what makes a view ``interesting''. A set of columns is
worth considering if it \emph{influences} the target: a selection on
the columns will affect the value of the target. For instance, the dimension
\texttt{Unemplyoment} is important because selecting cites with high
unemplyment leads to higher levels of crime. We can model this relationship
with \emph{mututal information}, which we present below.

The \emph{entropy} $H(\rv{X})$ of a variable $\rv{X}$ describes its
variability. If $\rv{X}$ has a constant value, then $H(\rv{X}) = 0$.
Oppositely, if $\rv{X}$ is highly unpredictable (e.g., $\rv{X}$ is
the outcome of flipping a perfectly balanced coin) then $H(\rv{X})$ is
maximal. Formally, if $\rv{X}$ is a discrete variable with sample space
$\Omega$, then we have:
\begin{equation}\label{eq:ent-disc}
    H(\rv{X}) = - \sum_{x\in \Omega} P(\rv{X} = x).\log{P(\rv{X} = x)}
\end{equation}
If $\rv{X}$ is continuous with density $p$, we define it as follows:
\begin{equation}\label{eq:ent-cont}
    H(\rv{X}) = - \int_{-\infty}^{+\infty} p(x).\log{p(x)}\mathrm{d}x
\end{equation}

We now describe how variables interact. If two two variables are dependent,
then conditioning (e.g., restricting the range of values) on one variable will
affect the other. In our example, cities with high unemployment have high
levels of crime.  Therefore, conditioning on the variable \texttt{Unemployment}
\emph{decreases the uncertainty} of the variable \texttt{Crime}. This causes a
loss in entropy, and the value of this loss is called \emph{mutual
information}.  Formally, if $\rv{X}$ and $\rv{T}$ are two random variables, the
expression $H(\rv{T} | \rv{X} = x)$ describes the entropy of $\rv{T}$
\emph{given $X = x$}. If we average this expression over all possible values of
$x$, we obtain the conditional entropy: $H(\rv{T}|\rv{X}) = \mathbb{E}_{x}
[H(\rv{T} | \rv{X} = x)]$. We define the mutual information $I$ as follows:
\begin{equation}\label{eq:mut-inf}
    I(\rv{X}; \rv{T}) = H(\rv{T}) - H(\rv{T} | \rv{X})
\end{equation}
The mutual information is the loss in entropy between $\rv{T}$ and
$\rv{T} | \rv{X}$. It is symmetic, and it is always positive or null. In our
study, a variable $\rv{X}$ is interesting if $I(\rv{X}; \rv{T})$ is as high as
possible.  We can generalize this notion to several variables: a set of
variables $\rv{X}_1, \ldots, \rv{X}_D$ is interesting they are \emph{jointly}
dependent to the target, e.g., $I(\rv{X}_1, \ldots, \rv{X}_D ; \rv{T})$ is
high.
\begin{definition}
Consider a view $V = \{X_1, \ldots, X_D\}$, and the target column $T$. We
define the \textbf{strength} of $V$ as follows: 
\begin{equation}\label{eq:strength}
    \sigma(V) = I(\rv{X}_1, \ldots, \rv{X}_D ; \rv{T})
\end{equation}
\end{definition} 
The stength is the central concept behind our study. We will spend most of this
paper describing how to detect strong sets of variables.

In practice, we cannot know the distributions of the variables $\rv{X}_n$, we
only have acess to the samples $X_n$. Therefore, we must compute \emph{estimates}
of the mutual informations $\hat{I}(X_n, T) \approx I(\rv{X}_n, \rv{T})$.  If
$\rv{X}_n$ is discrete, we simply set $\hat{P}(\rv{X}_n = x) \approx P(\rv{X}_n =
x)$ to be the proportion of tuples with $X_n=x$. We then ``plug'' this
estimator in Equations~\ref{eq:ent-disc} and~\ref{eq:mut-inf}.

Dealing with continuous variables is more difficult, for two reasons. First,
estimating the density function in Equation~\ref{eq:ent-cont} is costly,
especially with multivariate distributions.  Second, it is not clear how to
deal with mixed datasets, e.g. continuous and discrete dimensions, or discrete
target and continuous dimensions.  Therefore, Claude bins all the continuous
variables, and treats them like discrete dimensions.

\subsubsection{Points of Interest and Divergence}

We discussed how to recognize good views. In this section, we formalize our
notion of Point of Interest. A POI is a set of tuples for which the target has
an ``unusual'' distribution. Observe the two POIs in Figure~\ref{fig:overview}.
The distribution of \texttt{Crime} in these regions differs from the
distribution of \texttt{Crime} in the rest of the database.  In $POI_1$, the
distribution is skewed to the right, which indicates higher levels of crime. In
$POI_2$, it is skewed to the left. We will characterize this difference with the
\textbf{divergence}.

Consider a d-dimensional view $\{\rv{X}_1, \ldots, \rv{X}_D\}$. The set
$R\subset \Omega_1 \times \ldots \times \Omega_D$ represents a region in this
view.  The random variable $\rv{T}$ represents the target \emph{for the whole
database}.  The random variable $\big[ \rv{T} | (\rv{X}_1, \ldots, \rv{X}_D) \in R
\big] $ represents the target \emph{for the tuples within R}. From here on, we
will shorten this notation to $\rv{T} | R$. The region $R$ is a good point
of interest if $\rv{T} |R $ and $\rv{T}$ have large differences in
distribution.  We quantify this difference with the \emph{Kullback-Leibler
divergence} (KL).  The KL divergence measures the difference between two
probability distributions. It is null if the two distributions are similar, and
it grows when the distributions differ.  Formally, if $\mathcal{X}$ and
$\mathcal{Y}$ are two discrete random variables with the same sample space
$\Omega$, we have:
\begin{equation}\label{eq:KL_disc} 
    KL\big( \mathcal{X} \parallel \mathcal{Y} \big) = 
    \sum_{x\in \Omega} P(\rv{X} = x).\log{\frac{P(\rv{X} = x)}{P(\rv{Y} = x)}} 
\end{equation}
As Claude discretizes the continuous variables, we ignore the continuous case
(cf.  Section~\ref{sec:view-strength}). Our region $R$ is a good point of
interest if $KL ( \rv{T} \parallel \rv{T} |R )$ is as large as possible.
\begin{definition}
    Consider a set $R$ and the target variable $\rv{T}$. We define the
    \textbf{divergence} of $R$ as follows: 
\begin{equation}\label{eq:divergence}
    \delta(R) = KL \big( \rv{T} \parallel \rv{T} |R \big)
\end{equation}
\end{definition}

\begin{figure}[t!]
\centering
\includegraphics[width=0.6\columnwidth]{images/3Dtest}
\caption{Example of view with three discrete variables $V = \{X_1, X_2, X_3\}$.
The average divergence of the cells $\delta(\mathbf{x})$ equals the strength of
the view  $\sigma(V)$.}
\label{fig:binningexample}
\end{figure}
In fact, strength and divergence are closely connected.  Consider a view $V$.
If $V's$ dimensions are continuous, we bin them. If we compute the divergence
of each tuple and average the results, we obtain the $V$'s strength. We illustrate
this property with Figure~\ref{fig:binningexample}.  Therefore, strength and
divergence are ``two sides of the same coin''. We formalize this property it
with the lemma below.
\begin{lemma}
    If $V$ is a view with $d$ discrete variables and $\mathbf{x} \in \Omega_1
    \times \ldots \Omega_D$ is a tuple from this view, then:
    \begin{equation}\label{eq:coin}
        \sigma(V) = \mathbb{E}_{\mathbf{x}}  \big[ \delta(\{\mathbf{x}\}) \big]
    \end{equation}
\end{lemma}
\begin{proof}
    By definitions of the KL divergence, we have 
    $I(\mathbf{x} ; \rv{T})= \mathbb{E}_{\mathbf{x}}[ KL( T \parallel T |
    \{\mathbf{x}\} )]$.
    Substituting the left side with Equation~\ref{eq:strength}, and the right
    side with Equation~\ref{eq:divergence}, we obtain the lemma.
\end{proof}
Suppose that we obtained a view $V^*$ by discretizing a set of continuous
variables $V$. The average divergence of the bins equals the strength of $V^*$,
but not that of $V$. Luckily, these quantities converge as the bins get small.
\begin{lemma}
    The view $V$ is a set of continuous variables, $V^b$ is a discretized
    version of $V$ in which each variable is binned with bin size $b$, and
    $\mathbf{x}^b$ is a tuple from $V^b$. We have $\mathbb{E}_{\mathbf{x^b}}
    \big[ \delta(\{\mathbf{x^b}\}) \big] \to  \sigma(V)$ as $b \to 0$.
\end{lemma}
\begin{proof}
    Let the $D$-dimensional random vector $\mathbf{X}$ describe the
    (continuous) variables of $V$, and $\mathbf{X}^b$ describe the (discrete)
    variables of $V^b$. By generalizing Cover and Thomas, Theorem
    8.3.1~\cite{cover2012elements}, we infer that $H(\mathbf{X}^b) + D.\log{b}
    \to H(\mathbf{X})$ as $b \to 0$ .  Thus, using Equation~\ref{eq:mut-inf},
    we have $I(\mathbf{X}^b, \rv{T}) \to I(\mathbf{X}, \rv{T})$. We conclude
    that $\sigma(V^b) \to \sigma(V)$. We apply Equation~\ref{eq:coin} to obtain
    the lemma.
\end{proof}

\subsubsection{Problem Formulation}

We can now formulate our problem:
\begin{problem}
Consider a dataset $DB$, a target column $T$ and a triplet $(K, D, P)$. Find
the top $K$ strongest views with at most $D$ columns. For each of these
views, find the top $P$ divergent POIs.
\end{problem}
To solve this problem, Claude operates in two steps. First, it detects $K$
strong sets of columns.  We call this step \emph{column search}.  Then, it
extracts $P$ POI for each view. This is the \emph{POI dectection} step.















\section{Column selection}
\label{sec:colum}

In this section, we discuss how to detect strong sets of variables. From
here on, we assume that all the variables are discrete, and we use the
notations $D_n$ and $\rv{D}_n$ interchangeably - we can use the context to
distinguish database columns and random variables.

\subsection{Base algorithm}
\label{sec:base}
\begin{figure}[t!]
\centering
\includegraphics[width=\columnwidth]{images/beam-search}
\caption{Example of Beam Search, with $D=3$ and beam size $B=2$}
\label{pic:beam-search}
\end{figure}
Given a set of columns, our aim is to find the top $K$ views with at most $D$
columns. If our database includes $N$ dimensions, our search space contains
$\sum_{n \leq N} \binom{N}{n} = 2^N$ combinations, which is clearly
impractical. Therefore, we resort to a  greedy heuristic.

We operate in a level-wise manner, illustrated in Figure~\ref{pic:beam-search}.
To initialise the algorithm, we compute the strength of each variable
separately. We sort the candidates, and keep the top $B$ elements. We call this
set the \emph{beam}, greyed in the Figure. Then, we  generate new candidates by
appending each variable of the database to the variables in the beam. We obtain
views with two columns.  We keep the top $B$ view and discard the others, which
gives us a new beam. We repeat the procedure until the views in the beam
contain $D$ variables, or the views stop improving.
\begin{figure}[t!]
\centering
\includegraphics[width=0.8\columnwidth]{images/strength-jump}
\caption{Limit cases of the beam search strategy. The variables $X_1$ and
$X_2$ represent two dimensions. The symbol and color of the plots represent
the value of the target. }
\label{pic:strength-jump}
\end{figure}

Thanks to our beam strategy, we avoid exploring an exponentially large search
space. Instead, we compute the strengths of at most $N.B$ candidates at each
level. Our algorithm gets faster as the beam gets smaller.  Nevertheless, it is
not exact, and it may miss some good candidates if the beam is too tight. Let
us explain why. At each level of the algorithm, we discard the views which are
too weak to reach the top $B$ candidates. We assume that if a combination of
column is weak at level $i$, then it will be weak at all subsequent levels.
Unfortunately, this assumption does \emph{not} hold: we can form strong views by
combining weak columns. Therefore there are ``jumps'' in the search space.
Consider for instance the two classic scenarios pictured in Figure
\ref{pic:strength-jump}.  The dimensions $X_1$ and $X_2$ taken in isolation are
weak: we can infer no useful information about the target from either of them.
However, their combination is very interesting.  This observation is reflected
by the strength: the views $\{X_1\}$ and $\{X_2\}$ have a very poor strength,
but $\{X_1, X_2\}$ is an excellent candidate. In a beam search scenation, we
may discard $\{X_1\}$ and $\{X_2\}$ early because they have a low score.  This
would be a mistake, because we lose the opportunity to discover $\{X_1, X_2\}$.
To mitigate this effect, we set beam sizes larger than $K$.  During our
experiments, we obtained excellent results with $B \geq 2.K$
(cf.~\ref{sec:exp-view-selection}).


\subsection{Approximating the Strength of a View}
Our beam search algorithm proceeds in a greedy manner. It starts with simple
views, and adds variables one by one. To test if a variable $X_i$ is worth
adding to a view $V$, it computes the strength $\sigma(V \cup \{X_i\})$. If the
outcome is high enough, it will keep the candidate. If not, it will discard it.
In total, the algorithm must evaluate the strength of $B.N$ candidates for each
of the $D$ levels.  Unfortunately, computing the strength is very expensive -
it requires a group-by operation over the whole database. In this section, we present a
much faster approximation scheme.
 
So far, we have considered each level of our beam search in isolation: we have
\emph{not} made use of $\sigma(V)$ to compute the strength $\sigma(V \cup \{X_i\})$.
We have computed the score of each candidate ``from scratch''. Nonetheless, we
can derive a recursive formulation for the strength of a view. 
\begin{lemma}\label{lem:chain}
Consider a view $V = \{D_1, \ldots, D_i\}$, and a target $T$.
For any column $D_{i+1}$: 
\begin{equation}\label{eq:bonus}
\sigma(V \cup \{D_{i+1}\}) =  \sigma(V) + I(D_{i+1} ; T | D_1 , \ldots, D_i)
\end{equation}
\end{lemma}
\begin{proof}
This lemma is a consequence of the Mutual Information's chain rule,
Cover and Thomas, Theorem 2.5.2~\cite{cover2012elements}.
\end{proof}
This lemma describes how adding a column impacts the strength of a view.  For
any random variables $D_i,D_j,T$, the notation $I(D_j;T|D_i)$  expresses the
\emph{conditional mutual information}. The conditional mutual information is a
conditioned version of the mutual information: it describes the
dependency between $D_j$ and $T$ \emph{given restrictions on $D_i$}. To obtain
it, we compute the mutual information between $D_j$ and $T$ given all the
possible values of $D_i$, and average the results. Formally:
\begin{equation}\label{eq:condmutinfo}
    I(D_j;T|D_i) = \mathbb{E}_{x_i}\big[ I(D_j; T)| D_i = x_i)  \big]
\end{equation}
The influence of $D_i$ can go either way: it can weaken the dependency between
$D_j$ and $T$, or it can strengthen it. The conditional mutual information is
positive or null, and it is bounded by the entropy of $D_j$ and $T$. 

Unfortunately, we cannot directly exploit Lemma~\ref{lem:chain} in our
algorithm: computing $I(D_{i+1} ; T | D_1 , \ldots, D_i)$ is as expensive as
computing $\sigma(V \cup \{D_{j+1}\}) $ from scratch.  However, we can use an
approximation.  Recall that $V = \{D_1, \ldots, D_i\}$, we propose to use the
following scheme:
\begin{equation}\label{eq:approx}
\begin{split}
    \sigma(V \cup \{D_{i+1}\}) & = \sigma(V)   + I(D_{i+1} ; T | D_1, \ldots, D_{i})\\
                               & \approx \sigma(V) + I(D_{i+1} ; T | D_{i})
\end{split}
\end{equation}
The idea behind this approximation is naive: we assume that $I(D_{i+1} ; T |
D_0, \ldots, D_{n}) \approx I(D_{i+1} ; T | D_{i})$. We simply ignore the high
order dependencies. Thanks to this assumption, we can compute the strength of
our candidates much faster. 

\begin{figure}[t!]
\centering
\includegraphics[width=0.5\columnwidth]{images/codependency}
\caption{Example of co-dependency graph with 5 dimensions. To approximate the
strength of $V \cup \{X_5\}$, we add the weight of edge $(X_4, X_5)$ to V's
strength -  in this case 0.5.}
\label{pic:codependency}
\end{figure}
Our new algorithm operates in two steps, an offline step and an online step.
Offline, we compute the conditional mutual information  $ I(D_j ; T | D_i)$
between every pair of variable $(D_i, D_j)$. We call the resulting structure
\textbf{co-dependency graph}. In this graph, the vertices represent the
dimensions, and the edges represent the conditional mutual information. The
co-dependency graph is oriented and weighted.  Online, we run our beam search
as previously, but we use Equation~\ref{eq:approx} to compute the strength of
the new candidates.  Suppose that the beam contains a view $V_i= \{D_1, \ldots,
D_i\}$ at step $i$.  To evaluate the strength of a new candidate $V_i \cup
\{D_{i+1}\}$, we fetch the value of  $ I(D_{i+1} ; T | D_i)$ in the
codependency graph and add it to $V_i$'s strength. We illustrate this method in
Figure~\ref{pic:codependency}.  This algorithm is much faster because it spares
us expensive database operations. Previously, computing $\sigma(V_i \cup
\{D_{i+1}\})$ involved heavy groupings and aggregations on the whole database.
Now, we simply perform a lookup in a graph with $d$ edges.

Our approximation has a drawback: it depends on the order in which we include
the variables in the view. If we enrich a view by successively adding variables
$X_1$, $X_2$ then $X_3$, we obtain a different stength than if we incorporate
$X_3$, $X_2$ then $X_1$. In Equation~\ref{eq:approx}, we obtain different
approximations if we change the indexing of the dimensions $X_1, \ldots, X_i$.
For more robustness, we introduce a ``pessimistic'' variant of our
approximation:
\begin{equation}\label{eq:robustapprox}
    \sigma(V \cup \{D_{i+1}\}) 
    \approx \min_{n \in [1, i]} \sigma(V) + I(D_{i+1} ; T | D_{n})
\end{equation}
Instead of adding the strength $I(D_{i+1}; T | D_i)$, where $D_i$ is the last
variable inserted, we add $I(D_{i+1}; T | D_n)$, where $D_n$ is the variable
which weakens $D_{i+1}$ the most. We will use this version in the rest of the
paper.

In some use cases, obtaining approximations may not be enough.  To save some
accuracy, we can combine approximate and exact computations. In Claude's
implementation, we operate as follows. We use the exact version of the strength
to compute the first two levels of the beam search. Conveniently, we can derive
the co-dependency graph ``for free'' from these steps, noticing that $I(D_j; T
| D_i) = \sigma(\{D_i, D_j\}) - \sigma(\{D_i\})$. We then switch to approximate
computations. We revert to the exact computation for the final top $K$ ranking.

\subsection{Deduplication}
\label{sec:variety}
Our algorithm seeks strong views. In some cases however, it may be preferable
to have weaker views, but with more variety. To deal with these cases, we
introduce an optional \emph{deduplication step} in our algorithm.

We can decompose each level of beam search into two steps. First, we generate
new candidates, appending each of the $N$ columns of the database to each of
the $B$ views in the beam. We obtain $N.B$ candidates. Second, we evaluate the
candidates and keep the top $B$ strongest ones. The deduplication step occurs
between these two phases. The idea is to first reduce the $N.B$ candidates to $B'$
candidats with a deduplication algorithm. Then, we evaluate these $B'$
candidates, and keep the top $B$.


\section{Detecting Points Of Interest}
\label{sec:detec}

During this phase, we find the $r$ most divergent regions for each view.
Fortunatly, this task is an instance of a known Data Mining problem,
\emph{Subgroup Discovery} \cite{klosgen1996explora}\cite{wrobel1997algorithm}.
The aim of Subgroup Discovery is to identify sets for which the target value
maximizes a user-defined quality measure. To solve our problem, we instantiate
the quality measure with the divergence.

In principle, we could use any efficient algorithm from the Subgroup Discovery
litterature.  We propose to reuse the beam Search strategy from the previous
section, but to explore \emph{tuples} instead of columns. We bin the data in a
coarse manner and get a first set of $b$ cells. We then refine these cells with
a thinner binning, etc...

As mentioned in the Subgroup Discovery litterature \cite{van2011non}, our
divergence score has a drawback: it favours smaller groups.  Therefore, Beam
Search may converge very late or not at all.  A practical
solution is to alter the model to take the size into account. Let $R$
represents a range with cover $|R|$, and $|V|$ represent the number of tuples
in the view. We use the \emph{weighted} deviation $\delta_w(R) = |R|/|V| \times
\delta(R)$. This new score introduces a penalty for small POIs.


\section{Experiments}
\label{sec:experiments}

\begin{table}
    \centering
    \small
    \begin{tabular}{r c c c c} 
        \hline
        Dataset & Columns & Rows & \#Views & \#Variables\\
        \hline
        MuskMolecules & 167 & 6,600 & 22 & 18\\
        Crime & 128 & 1,996 & 20 & 17\\
        BreastCancer & 34 & 200 & 10 & 13\\
        PenDigits & 17 & 7,496 & 9 & 10\\
        BankMarketing & 17 & 45,213 & 11& 8\\
        LetterRecog & 16 & 20,000 & 10 & 12\\
        USCensus & 14 & 32,578 & 10 & 7\\
        MAGICTelescope & 11 & 19,022 & 1 & 10\\
        \hline
    \end{tabular}
    \caption{Characteristics of the datasets. The last two columns are used for
    comparison with 4S, cf. Section~\ref{sec:exp-view-selection}.}
    \label{tab:datasets}
\end{table}
We now present our experimental results. All our experiments are based on 8
datasets from the UCI Repository, described in Table~\ref{tab:datasets}. The
files are freely available online\footnote{archive.ics.uci.edu/ml/}.

\subsection{Detailed Example: Crimes in the US}
\label{sec:crime}

\begin{table}[t]
  \centering
  \small
  \rowcolors{2}{gray!25}{white}
  \begin{tabulary}{\columnwidth}{L c}
    \hline
    View & Score (normalized)\\
    \hline
    Police.Overtime, Pct.Vacant.Boarded, Pct.Race.White & 0.51\\
    Pct.Families.2.Parents, Pct.Race.White, Police.Requests.Per.Officer & 0.49\\
    Pct.Police.White, Pct.Police.Minority, Pct.Vacant.House.Boarded& 0.37\\
    Pct.Empl.Profes.Services, Pct.Empl.Manual, Pct.Police.On.Patrol & 0.37\\
    Pct.Retired, Pct.Use.Public.Transports, Pct.Police.On.Patrol& 0.35 \\
    Pct.Recently.Moved, Population.Density, Police.Cars & 0.34 \\
    \hline
\end{tabulary}
    \caption{Example of views generated by Claude for the US Crime dataset.}
    \label{tab:crime_views}
\end{table}
\begin{figure}[t!]
    \centering
    \begin{subfigure}[b]{0.4\columnwidth}
    \includegraphics[width=\textwidth]{plots/crime1}
    \end{subfigure}
    \begin{subfigure}[b]{0.4\columnwidth}
    \includegraphics[width=\textwidth]{plots/crime2}
    \end{subfigure}

    \begin{subfigure}[b]{0.4\columnwidth}
    \includegraphics[width=\textwidth]{plots/crime3}
    \end{subfigure}
    \begin{subfigure}[b]{0.4\columnwidth}
    \includegraphics[width=\textwidth]{plots/crime4}
    \end{subfigure}
    
    \begin{subfigure}[b]{0.4\columnwidth}
    \includegraphics[width=\textwidth]{plots/crime5}
    \end{subfigure}
    \begin{subfigure}[b]{0.4\columnwidth}
    \includegraphics[width=\textwidth]{plots/crime6}
    \end{subfigure}
\caption{Heatmaps of the US Crime Dataset, based on Claude's output. Each box
represents a Point of Interest.}
\label{pic:crime_charts}
\end{figure}

In this section, we showcase Claude with a real-life example: we analyze the
Communities and Crime dataset form the UCI
repository\footnote{archive.ics.uci.edu/ml/datasets/Communities+and+Crime}.
Our aim is to understand which US cities are subject to violent crimes. Our
database compiles crime data and socio-economic indicators about 1994
communities, with a total of 128 variables. The data comes mostly from the
90's, and it was provided by US government sources - among others, the 1990 US
census and the 1995 FBI Uniform Crime Report. All the variables are
normalized to have a minimum of 0 and a maximum of 100.

We generated $K=100$ explanations with up to $D=3$ dimensions, both with and
without deduplication. We present a selection of explanations in
Table~\ref{tab:crime_views}, along with 2-dimension heatmaps in
Figure~\ref{pic:crime_charts}. Observe that strong views have a visual
signature: in the top two maps, the blue and red areas are neatly separated. In
the bottom two views, the distinction is less clear.

The first view of Table \ref{tab:crime_views} is the best one we found:
\texttt{Police. Overtime, Pct.Race.White, Pct.Vacant.Boarded}. It has a score
of 0.51. which means that these three variables contain 51\% of the target's
information. The columns \texttt{Police. Overtime} and \texttt{Pct.White.Race}
respectively describe the average time overworked by the police and the
percentage of caucasian population. The third variable,
\texttt{Pct.Vacant.Boarded} was surprising to us: it describes the
percentage of vacant houses which are boarded up.  How does this relate to
crime? We could assume that boarded houses are associated with long term
abandon, and thus, poverty. The top-left plot of Figure \ref{pic:crime_charts}
shows the relation between race, boarded houses and crime. Observe that the
variables complement each other: a high proportion of caucasians may or may not
lead to low crime. However, a high proportion of caucasians \emph{combined
with} a low rate of boarded house correspond to safe areas, while little
caucasians and many boarded houses correspond to more violent communities.

Our second view shows that cities with more monoparental families tend to be
more violent: the correlation is clearly visible, and both POIs point to the
bottom of the chart. However, close inspection also reveals surpsises: a few
communities have a relatively high number of two-parents families, but also
high indicators of police requests and crime (in the top right corner of the
chart). Manual queries reveals that many of these cities are located in the
suburbs of Los Angeles, and contain a majority of Hispanics. Does this explain
the peculiarity? We leave this question open for future investigations. We see
that some findings come from the recommendations directly while others are
serendipitous.  But in both cases, Claude lets us discover ``nuggets'' with
little prior knowledge and few assumptions. 

\subsection{View Strength and Prediction}
\label{sec:view-strengh}

\begin{figure}[t!]
\centering
\includegraphics[width=\columnwidth]{plots/compare-strength-f1}
\caption{Strength vs. Classification accuracy for 500 random views. The blue
and red lines were obtained by linear regression.}
\label{pic:strength-vs-f1}
\end{figure}
In this section, we show experimentally that our notion of view strength
``works'', e.g. that strong views effectively provide information about the
target column. To verify this assumption, we simulate users with statistical
classifiers. Consider a view $V$ over a database. If a classifier can predict
the value of the target from $V$'s columns, then $V$ is informative.
Oppositely, if the classifier fails, then $V$ is potentially uninteresting. In
other words, we should observe a positive correlation between views strength
and classification accuracy.

We now detail our experiment. We chose three datasets from the UCI repository.
For each dataset, we generated 500 random views and measured their strengths.
We then trained classifiers on each of these views, and measured their
performance. We report the results in Figure \ref{pic:strength-vs-f1}. We chose
two classification algorithms: Naive Bayes, and 5-Nearest Neighbors.  We chose
those because they contain no built-in mechanism to filter out irrelevant
variables (as opposed to, e.g., decision trees). We measure the classification
performance with 5-fold validation, to avoid the effects of overfitting.

In all three cases, we observe a positive correlation between the strengths of
the views and the accuracy of the predicictions. We confirm these observations
with statistical tests: the coefficients of determination ($R^2$) vary between
0.11 and 0.84, which indicates the presence of a trend (despite some
variablity). Furthermore, the p-values associated to the coefficients are all
under $10^{-3}$, this gives us excellent confidence that the strength
influences positively the prediction accuracy. In conclusion, strong views are
indeed more instructive.


\subsection{View Selection}
\label{sec:exp-view-selection}

We now evalute Claude's output and runtime in detail. In this section, we
verify if Claude's algorithm produces good views in a short amount of time. To
do so, we compare it to four methods, three of which come from the machine
learning litterature.  Our first baseline, \texttt{Exact}, is similar to
Claude, but we removed the approximation scheme presented in \ref{sec:approx} -
instead we compute the exact mutual information between the views and the
target. The method should be slower, but more accurate. 

The second algorithm, \texttt{Clique}, is a top-down approach inspired by
recent work on pattern mining~\cite{xie2010max}. The idea is to build a graph
where each vertex $i$ represents a column $D_i$, and each edge $(i,j)$
represents the view $\{D_i, D_j\}$. We then simplify the graph: we keep the
edges associated with the top $B$ strongest views, and we eliminate all the
others. To detect strong views with $D >2$ variables, we seek cliques in this
degenerated graph. We used the \texttt{igraph} package from R. We expect this
algorithm to be very fast, but less accurate.

The third method, \texttt{Wrap 5-NN}, is a classic feature selection
algorithm~\cite{guyon2003introduction}. The idea is to train a 5-Nearest
Neighbour classifier with increasingly large sets of variables. We first test
each variable separately, and keep the column which led to the best prediction.
Then we keep adding variables in a Breadth-First manner, until the
quality of the predictions stops increasing or we reach $n$ variables. Our
implementation is based on the \texttt{class} package from R.  We modified the
original algorithm to maintain and update $q$ distinct sets of variables
instead of just one. We chose the nearest neighbour algorithm because it is
fast, and it gave us good performance, as shown in \ref{sec:view-strengh}. We
expect this algorithm to be very slow, but close to optimal.

Finally, the last method, \texttt{4S} is a state-of-the-art subspace search
method from the unsupervised learning literature~\cite{nguyen20134s}. The aim
of the algorithm is to detect ``interesting'' subspaces in large databases,
independantly of a target variable. To do so, it seeks groups of variables
which are mutually correlated, with sketches and graph-based techniques.  We
used the author's implementation, written in Java. We expect the algorithm to
be very fast and reasonably accurate.

We use 8 public datasets, presented in Section \ref{sec:experiments}. For a
fair comparison, we must ensure that each algorithm generates the same number
of views ($K$) with the same number of variables ($D$).  However, we have no
way to specify these parameters a priori with 4S, because the algorithm has a
built-in mechanism to pick optimal values. Therefore, we run 4S first on each
dataset, we let it chose $K$ and $D$, and we use these values for the remaining
algorithms.  We report the obtained parameters in Table~\ref{tab:datasets}.

We implemented Claude in R, except for some information theory primitives
written in C. For practical reasons, we interupted all the experiments which
lasted more than 1 hour. Our test system is based on a 3.40 GHz Intel(R)
Core(TM) i7-2600 processor. It is equipped with 16 GB RAM, but the Java heap
space is limited to 8 GB. The operating system is Fedora 16. 

\begin{figure*}[t!]
\centering
\includegraphics[width=2\columnwidth]{plots/view-scores}
\caption{Performance of the View Selection algorithms. For each data set, we
    generate $q$ views, train a 5-NN classifier over the columns of each view
    and report the classification accuracy (F1, 5-fold cross validation). The
    points represent median scores, the bars represent the lowest and greatest
    scores.} 
\label{pic:column-select-score}
\end{figure*}
\begin{figure*}[t!]
\centering
\includegraphics[width=2\columnwidth]{plots/view-times}
\caption{Execution time of the View Selection algorithms. A \texttt{X} symbol
indicates that the experiment did not finish within 3,600 seconds.}
\label{pic:column-select-time}
\end{figure*}

\textbf{Accuracy.} In Figure \ref{pic:column-select-score}, we compare the
quality of the views returned by each algorithm. For each competitor, we
generate $q$ views with $n$ variables, train a classifier on each view and
measure the quality of the predictions. For the classification, we use both
Naive Bayes and 5-Nearest Neighbors, and report the highest score.  We measure
accuracy with the F1 score on 5-fold cross validation; higher is better.

The method \texttt{Wrap 5-NN} comes first for all the datasets on which it
completed. This is not surprising since the algorithm optimizes exactly what we
measure: \texttt{Wrap 5-NN} is our ``gold standard''. Our two algorithms,
\texttt{Claude} and \texttt{Exhaustive}, come very close. This indicates that
both algorithms find good views, and that our approximation scheme works
correctly.  The algorithms \texttt{4S} and \texttt{Clique} come much lower. As
\texttt{4S} is completely unsupervised, we cannot expect it to perform as well
as a the other approaches. The assumptions behind \texttt{Clique} are
apparently too naive.

\textbf{Runtime.} Figure~\ref{pic:column-select-time} shows the runtime of our
experiments. The algorithms \texttt{Exact} and \texttt{Wrap 5-NN} are orders of
magnitude slower than the other approaches. The remaining three approaches are
comparable: depending on the datasets, either \texttt{Clique} or \texttt{4S}
come first. Claude comes first for \texttt{MuskMolecules}, and close
second for all the other datasets. We conclude that Claude is compairable to
state-of-the-art algorithms in terms of runtime, but it generates better views.

\begin{figure}[t!]
\centering
\includegraphics[width=\columnwidth]{plots/view-vary-beam}
\caption{Impact of the beam size on the execution time and view strength. For
each dataset, we generated 25 views with beam size 25, 50, 100 and 250. The
points represent the medium scores, the bars represent the lowest and greatest
scores.}
\label{pic:view-beam}
\end{figure}
\begin{figure}[t!]
\centering
\includegraphics[width=\columnwidth]{plots/view-vary-diversification}
\caption{Impact of the deduplication. The y-axis presents the number of
distinct variables used in the top 25 views.}
\label{pic:view-diversification}
\end{figure}
\textbf{Impact of the beam size.} Figure~\ref{pic:view-beam} shows the impact
of the beam size $B$ on Claude's performance, for 4 databases. To obain these
plots, we ran Claude with $K=25$ and $D=5$, and varied $B$ between 25 and 250.
We observe that smaller beams lead to lower execution times, while larger beam
lead to stronger views. However, the heuristic converges fast: we observe
little to no improvement for $B$ greater than 50.

\textbf{Impact of the deduplication.} We show the impact of our deduplication
strategy in Figure \ref{pic:view-diversification}.  We ran Claude with $K=25$
and $D=5$, and varied the level of deduplication. To quantify the diversity of
the views, we report the number of distinct variables used. We observe that the
strategy works in all four cases, but with different levels of efficiency. In
the \texttt{LetterRecog} case, our strategy increases the number of variables
by almost 30\%. The effect is much lighter on \texttt{MuskMolecules}.

\subsection{POI Detection}
\label{sec:exp-poi}

\begin{figure*}[t!]
\centering
\includegraphics[width=2\columnwidth]{plots/POI-score}
\caption{Quality of the Point of Interests. For each view, we detect $P=10$
POIs. The
    points represent median scores, the bars represent the lowest and greatest
    scores.}
\label{pic:POI-quali}
\end{figure*}
\begin{figure*}[t!]
\centering
\includegraphics[width=2\columnwidth]{plots/POI-timing}
\caption{Execution time of the POI detection. A \texttt{X} symbol
indicates that the experiment did not finish within 7,200 seconds.}
\label{pic:POI-time}
\end{figure*}

In this section, we evaluate Claude's POI detection strategy. We compare two
approaches. The first approach is the algorithm presented in the paper: first
we search $K$ views, then we return $P$ POIs per view. The second approach,
\texttt{FullSpace}, is the method used in much of the recent Subgroup Discovery
litterature \cite{van2011non, duivesteijn2010subgroup}.  The idea is to apply
Beam Search on the whole database directly. Instead of seeking $P$ POIs in $K$
projections, we seek $K.P$ selections from the full column space; we skip the
view selection step. We use the same datasets as previoulsy. Our default
parameters are $K=25$, $D=5$, $B=50$ and $P=10$. To set the beam size, we use a rule of
thumb: $B_{POI} = 2.k$ ($B_{POI}$ is the beam used for POI detection,
not for view search). In order to gather sufficient data, we raise our time
limit to 2 hours. 

Figure \ref{pic:POI-quali} compares the quality of the POIs found by both
algorithms. The strategy \texttt{FullSpace} gives slightly better results on
\texttt{Crime} and \texttt{PenDigits}, but the difference is close to null. The
scores are similar on all the other datasets. We conclude that Claude's POIs
are very close to those found by a state-of-the-art Subgroup Discovery approach.
Figure~\ref{pic:POI-time} compares the runtimes of both approaches.  We observe
that Claude is much faster than \texttt{FullSpace}. The difference grows with
the number of columns: the runtimes are almost similar for datasets with few
columns (\texttt{MAGICTelescope}), but \texttt{Claude} is considerably faster
for larger databases (more than an order of magnitude difference for
\texttt{MuskMolecules}). Although this was not our initial aim, decoupling view
search and POI extraction allows us to find subgroups faster in high dimension
datasets.

\section{Related Work}
Our work is inspired by both database research and machine learning. On the
database side, Claude is related to the \emph{query recommendation} problem:
how to automatically generate interesting SQL queries for a given database? On
the machine learning side, our work is based on \emph{feature selection}, e.g.,
how to select variables for a given inference problem, and \emph{subgroup
discovery}, e.g., how to find the tuples which maximize a particular score
function.

\textbf{Query Recommendation.} We identify two types of recommendation systems:
\emph{human-driven} approaches and \emph{data-driven} approaches. Human-driven
systems learn from user feedback. For instance, Chatzopoulou et al. propose to
make recommendations from query logs, similarly to search engines
\cite{chatzopoulou2009query}. More recently, authors have proposed interactive
exploration systems, where a recommendation system ``guides'' the user through
the database. In Explore-by-Example, the system infers queries from examples
provided by the user \cite{dimitriadou2014explore}. With Charles, the engine
decomposes large queries into smaller ones \cite{sellam2013meet}. Sarawagi's
method builds a maximum entropy model over the database from the user's history 
\cite{sarawagi2000user}. Bonifati et al. propose a similar method to recomend
joins \cite{bonifati2014interactive}.  Claude competes with neither of these
approaches, since it uses the content of the database only.

Our work is closer data-driven data recommendation, in which the system makes
recommendations based on the content of the data. The general idea is to build
a statistical model of the database, and find regions which behave
unexpectedly. Sarawagi et al. have published seminal work on this topic for
OLAP cubes \cite{sarawagi1998discovery}. Their system highlights sequences of
drill-in operations which lead to ``surprising data'', e.g., tuples which
differ from their neighbourhood. It requires that the data is organized in an
OLAP cube (with hierchical dimensions), it supposes that the users know which
variables to use, and it seeks thin-grained deviations. Oppositely, our system
uses regular tables, it recommends views (not only selections) and it seeks
large trends. More recently, Dash et al. have proposed a method to reveal
surprising subsets in a faceted search context \cite{dash2008dynamic}. This
method is related to Claude, but it targets document search, it does not
recommend views.

\textbf{Projection Search.} cf. data viz: Grand Tour, scagnostics, etc...


\textbf{Feature Selection, Subspace Search.} Chosing which variables to use for
classification or regression is a crucial problem, for which dozens of methods
were proposed \cite{guyon2003introduction}. Similarly to Claude, some of these
methods rely on mutual information~\cite{peng2005feature} Nevertheless, the
objective is different. A feature selection algorithm seeks \emph{one} set of
variables on which a statistical predictor will perform optimally. Claude seeks
several, small sets of variables, simple enough to be interepreted by a humans.
Claude is halfway between inference and exploration.

On the unsupervised learning side, our work is close to subspace search. The
idea is detect subspaces where the data is clustered distinctly
\cite{keller2012hics,nguyen20134s}. We compare Claude to state-of-the-art
methods in our Experiments section.

\textbf{Subgroup Discovery.} Our work is very close to subgroup discovery
\cite{klosgen1996explora, wrobel1997algorithm, van2011non}. We discuss the
connections and compare approaches in Sections~\ref{sec:detec}
and~\ref{sec:exp-poi} respectively.

\section{Conclusion}

Future work: Causation models, synergy with visualisation


\bibliographystyle{abbrv}
\bibliography{TurboGroups}
\balancecolumns

\end{document}
